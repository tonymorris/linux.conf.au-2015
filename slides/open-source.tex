\begin{frame}[fragile]
\frametitle{Parametricity and open-source}
\begin{block}{my open-source goals}
\begin{itemize}
  \item can fix bugs independently of the possibility of creating more
  \item can introduce features without adversely affecting others
  \item can have hundreds of projects requiring zero maintenance
  \item \textbf{avoid endless tail-chasing that prevails in corporate dev}
\end{itemize}
\end{block}
\end{frame}

\begin{frame}[fragile]
\frametitle{Parametricity and open-source}
\begin{block}{Department of Defense 29 January 2016}
\begin{quotation}
The Marine Corps' F-35B aircraft are being delivered with Block 2B software, which Gilmore said has "hundreds of unresolved deficiencies." And those problems have compounded in Block 3F software. That's because the first round of Block 3 was created by "re-hosting the immature Block 2B software…into new processors to create Block 3i," the initial release for the code, Gilmore noted. This led to "avionics instabilities and other new problems, resulting in poor performance during developmental testing."
\end{quotation}
\end{block}
\end{frame}


\begin{frame}[fragile]
\frametitle{Parametricity and open-source}
\begin{block}{goals}
The value of parametricity is too high to forgo in open-source development. And for what possible benefit?
\end{block}
\end{frame}

\begin{frame}[fragile]
\frametitle{Parametricity and open-source}
\begin{block}{goals}
\begin{itemize}
  \item what's it like for you haskell programmers in the ivory tower?
  \item why are you so averse to programming language environment X?
\end{itemize}
\end{block}
\end{frame}
