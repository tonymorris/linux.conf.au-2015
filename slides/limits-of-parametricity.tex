\begin{frame}[fragile]
\frametitle{The Limits of Parametricity}
\begin{lstlisting}
<A> List<A> function(List<A>)
\end{lstlisting}
OK, so we know that all elements in the result appear in the input
\begin{itemize}
  \item convince yourself of this fact
  \item but how do we narrow it down?
  \item how do we rule out all possibilities for the type but one?
  \item how do we unambiguously determine what the function does?
\end{itemize}
\end{frame}

\begin{frame}[fragile]
\frametitle{The Limits of Parametricity}
\begin{block}{what does this function do?}
\lstinputlisting[style=haskell]{source/reverse-with-tests.hs}
\end{block}
\end{frame}

\begin{frame}[fragile]
\frametitle{Once-inhabitance}
\begin{block}{sometimes tests are unnecessary}
\begin{lstlisting}[style=haskell]
f :: a -> a
\end{lstlisting}
\end{block}
\end{frame}

\begin{frame}[fragile]
\frametitle{Once-inhabitance}
\begin{block}{sometimes tests are unnecessary}
\begin{lstlisting}[style=haskell]
g :: Functor f => y -> f x -> f y
\end{lstlisting}
\end{block}
\begin{lstlisting}[style=haskell]
> g "hi" [1,2,3]
["hi","hi","hi"]
\end{lstlisting}
\end{frame}

\begin{frame}[fragile]
\frametitle{Parametricity}
\begin{block}{non-trivial example}
\begin{lstlisting}[style=haskell]
both ::
  (Applicative f, Bitraversable r) =>
  (a -> f b) -> r a a -> f (r b b)
\end{lstlisting}
\end{block}
\tiny
\begin{itemize}
  \item<1-> This function can only \lstinline{bitraverse} \footnote{(and derivatives)} on \lstinline{(`r`)}
  \begin{itemize}
    \item \tiny{will work with \lstinline{Either} at call site}
    \item \tiny{will work with \lstinline{(,)} at call site}
    \item \tiny{will work with \lstinline{Const} at call site}
    \item \tiny{\textbf{but \lstinline{both} cannot do anything specific to these data types}}
  \end{itemize}
  \item<2-> This function can only \lstinline{(<*>)} and \lstinline{pure} on \lstinline{(`f`)}
  \begin{itemize}
    \item \tiny{will work with \lstinline{Maybe} at call site}
    \item \tiny{will work with \lstinline{IO} at call site}
    \item \tiny{e.g. call site can open network connections using \lstinline{both}}
    \item \tiny{\textbf{however \lstinline{both} definitely does not open any network connections itself}}
  \end{itemize}
  \item<3-> \lstinline{(`a`)} and \lstinline{(`b`)} might be anything
  \begin{itemize}
    \item \tiny{may be \lstinline{Int} at call site}
    \item \tiny{may be \lstinline{String} at call site}
    \item \tiny{\textbf{however \lstinline{both} definitely does not perform any \lstinline{Int}-specific operations}}
  \end{itemize}
\end{itemize}
\end{frame}

\begin{frame}[fragile]
\frametitle{Parametricity}
\begin{block}{and on it goes}
\begin{lstlisting}[style=haskell]
(<.) ::
  Indexable i p =>
  (Indexed i s t -> r) -> ((a -> b) -> s -> t) -> p a b -> r
\end{lstlisting}
\end{block}
\end{frame}
